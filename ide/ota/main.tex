Here we show how to program ESP32 through OTA using platformio.
The below link guides to the codes required for implementation of ESP32 through OTA  
\begin{lstlisting}
https://github.com/pradyumna963/afw/tree/main/ide/ota
\end{lstlisting}
\begin{enumerate}[label=\arabic*.,ref=\theenumi]
\item Follow the steps 1 and 2 given in section 1.2.
\item Open the main.cpp file by excecting the below command in termux and change the SSID and password mentioned in the main code.
\begin{lstlisting}
nvim /ide/ota/blink/src/main
\end{lstlisting}
\item In termux excecte the following to generate the bin file:
\begin{lstlisting}
cd ide/ota/blink
pio run
\end{lstlisting}
\item Create a folder in AruinoDroid/precompiled dirctory and copy paste the bin file to this folder by excecting the following commands:
\begin{lstlisting}
mkdir /sdcard/Arduinodroid/precompiled/blink_ota
cp ls .pio/build/esp32doit-devkit-v1/firmware.bin /sdcard/Arduinodroid/precompiled/blink_ota
\end{lstlisting}
\item For flashing the bin files, open ArduinoDroid
\begin{lstlisting}
Actions->Upload->Upload Precompiled
\end{lstlisting}
then select 
\begin{lstlisting}
blink_ota->firmware.bin
\end{lstlisting}
for uploading the bin bile to the ESP32.
\item After the uploading is finished you will get the following in the terminal.
\begin{lstlisting}
Error: open failed: ENOENT (No such file or directory)
\end{lstlisting} 
Disconnect the power supply and reconnect it, then the inbuilt led will start blink. The ESP32 will also be connected to the wifi network that is mntioned in the code.
\item For the next program that needs to be flashed you can upload the code through OTA by excecuting the following commands in termux following commands
\begin{lstlisting}
cd -
cd ide/ota/sevenseg/static
pio run -t upload --upload-port 192.168.0.0 #use the IP address of the ESP32
\end{lstlisting}
\end{enumerate}
